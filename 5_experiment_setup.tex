\subsection{Systems}
Our experiments are conducted on two NVIDIA GPUs: a V100 GPU and a P100 GPU in six systems of different memory size and interconnection type. Specific details of these GPUs can be found in Table \ref{tab:gpu-cpu-systems}. 
The P100 is a Pascal based GPU and the V100 is a Volta based GPU. 
Both the V100 and P100 have two form factors: PCI-E and SXM. PCI-E is the defacto form factor that most consumer GPUs are manufactured. The SXM form factor is equivalent to placing a GPU on a board. The SXM form factor also has multiple NVLink channels allowing GPUs to communicate with other multiple GPUs concurrently. For both the V100 and P100, the SXM form factor GPU is known for outperforming its PCI-E counterpart due to increased frequency and power consumption.

% Each GPU was connected to a different CPU processor. We do not report the CPUs used in our experiments as they are not of significance; specifically, we do not time the transfer to and from the GPU. Instead to ensure that execution times capture the time it takes to run an analytic, we time only sections that are entirely on the GPU side and are part of the analytic procedure.
% Euna: CPU info added to the table

The effect of CPUs used in our experiments are not significant; specifically, we do not time the transfer to and from the GPU. To ensure that execution times capture the time that it takes to run an analytic, we time only sections that are entirely on the GPU side and are part of the analytic procedure.
% \begin{scriptsize}
% \begin{table*}[t]
% %\vspace*{-0.1 cm}
% \caption{%GPU P100 and V100 are used in our experiments.
% Each GPU used for the experiments tested for two different form factors: PCI-E and SXM-2.% Each of the form factors has a different power limitation. 
% }
% \centering
% %\vspace*{-0.1 cm}
% \begin{tabular}{|c|c|c|c|c|c|c|c|c|c|} \hline 
% %\begin{tabular}{ |p{1cm}|p{1cm}|p{1cm}|p{1cm}|p{1cm}|p{1cm}|p{1cm}|p{1cm}|p{1cm}| } \hline 
% %\begin{tabular}{ |p{0.5cm}|p{0.4cm}|p{0.4cm}|p{0.5cm}|p{0.5cm}|p{0.5cm}|p{0.5cm}|p{0.7cm}|p{0.6cm}| } \hline 
% Processor & Micro arch. & SM & SP (/SM) & Total SPs & DRAM Size & DRAM Type & Form factor & Power  \\ \hline
% P100  & Pascal     & 56    & 64 & 3854 & 16GB    & HBM2 & PCI-E / SXM-2 & 250W / 300W\\  \hline 
% % GPU-CUDA     & P100  & Pascal     & 56    & 64 & 3854 & 16GB    & HBM2 & SXM-2 & 250W\\  \hline 
% V100  & Volta     & 80    & 64 & 5120 & 16GB    & HBM2 & PCI-E / SXM-2 & 250W / 300W \\  \hline 
% % GPU-CUDA     & V100  & Volta     & 80    & 64 & 5120 & 16GB    & HBM2 & SXM-2 & 300W \\  \hline 
% V100  & Volta     & 80    & 64 & 5120 & 32GB    & HBM2 & PCI-E / SXM-2 & 250W / 300W \\  \hline 
% % GPU-CUDA     & V100  & Volta     & 80    & 64 & 5120 & 32GB    & HBM2 & SXM-2 & 300W \\  \hline 
% \end{tabular}
% \vspace*{0.2 cm}
% \label{tab:gpu-cpu-systems}
% \end{table*}
% %[Euna] adding bandwidth, flops info, cuda version....
% \end{scriptsize}

\begin{scriptsize}
\begin{table*}[htbp]
\begin{threeparttable}
 \caption{[TBD] Description of six GPU systems used for the experiments}
 \label{tab:gpu-cpu-systems}
 \centering
\begin{tabular}{qccccccccccc}
     \toprule
    % Processor & Micro arch. & SM & SP (/SM) & Total SPs & DRAM Size & DRAM Type & Form factor & Power  
    \multicolumn{1}{c}{System$^a$} &
    \multicolumn{1}{c}{CPU} & 
    \multicolumn{1}{c}{Form factor} &
    \multicolumn{1}{c}{Processor} & 
    \multicolumn{1}{c}{M.Arch.} & 
    \multicolumn{1}{c}{SMs} & 
    \multicolumn{1}{c}{SP/SM} & 
    \multicolumn{1}{c}{Total SPs} &
    \multicolumn{1}{c}{DRAM Size} & 
    \multicolumn{1}{c}{DRAM Type} & 
    \multicolumn{1}{c}{Power} \\
\midrule
    P100\_16GB\_PCI & 2.30GHz$^b$ & PCI-E & P100 & Pascal & 56 & 64 & 3854 & 16GB & HBM2 & 250W \\ 
    P100\_16GB\_SXM & (TBD) & SXM-2 & P100 & Pascal & 56 & 64 & 3854 & 16GB & HBM2 & 300W \\ 
    V100\_16GB\_PCI & 2.30GHz$^b$ & PCI-E & V100 & Volta  & 80 & 64 & 5120 & 32GB & HBM2 & 250W \\ 
    V100\_16GB\_SXM & 2.20GHz$^c$ & SXM-2 & V100 & Volta  & 80 & 64 & 5120 & 32GB& HBM2 &  300W \\ 
    V100\_32GB\_PCI & 2.30GHz$^b$ & PCI-E & V100 & Volta  & 80 & 64 & 5120 & 32GB& HBM2 &  250W \\ 
    V100\_32GB\_SXM & 2.20GHz$^c$ & SXM-2 & V100 & Volta  & 80 & 64 & 5120 & 32GB& HBM2 &  300W \\
     \bottomrule
   \end{tabular}
     \begin{tablenotes}
      \tiny
      \item {
      $^a$ Device driver version 418.67 and operating system CentOS Linux 7 for all systems, 
      $^b$ Intel(R) Xeon(R) CPU E5-2698 v3 @ 2.30GHz with cache 40960KB, 
      $^c$ Intel(R) Xeon(R) CPU E5-2698 v4 @ 2.20GHz with cache 51200KB}
    \end{tablenotes}
  \end{threeparttable}
 \end{table*}
\end{scriptsize}
%



\subsection{Graphs}
Details of the graphs used in our experiments can be found in the upper part of Table \ref{tab:graphs}. This table covers the number of vertices, edges, and average degree of each graph. For each network, the table also denotes if that network is stored as a directed or undirected graph. Undirected graphs store both directions of each edge. The experiments are executed for both directed and undirected versions of each graph. Thus, if a graph is given as a directed graph, for its undirected version the inverse direction of each edge is added. If that edge already exists, then that duplicate edge is not included. Different frameworks load directed and undirected graphs using different methods; Most of graph frameworks support graph loading features with user parameters. Loading graphs as the original structure(undirected graphs as undirected, directed graph as directed) is not complicated and explained above. We found that Hornet generates a directed graph from an undirected graph by taking the first read edge and ignoring the second edge in the opposite direction while Gunrock ignores "--directed" for loading undirected graphs. 
%graph data
% \begin{table}[t]
% \begin{footnotesize}
% \begin{center}
% \caption{Graph data used for experiments. $|E|$ refers to directed edges. Networks are sorted based on the number of edges.}
% \begin{tabular}{|c|l|r|r|r|r|r|}
% \hline
% \multicolumn{ 1}{|c|}{} & \multicolumn{1}{c|}{Name} & \multicolumn{1}{c|}{$|V|$} & \multicolumn{1}{c|}{$|E|$} & \multicolumn{1}{c|}{Size} & \multicolumn{1}{c|}{Structure} & \multicolumn{1}{c|}{Ave.Deg} \\ \cline{ 2- 7}
% \multicolumn{ 1}{|c|}{Real} & cit\_Patents & 3,774,768 & 16,518,948 & 249 MB & Directed & 24 \\ \cline{ 2- 7}
% \multicolumn{ 1}{|c|}{} & Soc-LiveJournal1 & 4,847,571 & 68,993,773 & 964 MB & Directed & 14 \\ \cline{ 2- 7}
% \multicolumn{ 1}{|c|}{} & Soc-twitter-2010 & 18,520,486 & 298,113,762 & 4,809 MB & Undirected & 16 \\ \cline{ 2- 7}
% \multicolumn{ 1}{|c|}{} & uk-2002 & 21,297,772 & 530,051,618 & 4,977 MB & Directed & 25 \\ \cline{ 2- 7}
% \multicolumn{ 1}{|c|}{} & uk-2005 & 39,459,925 & 936,364,282 & 15,689 MB & Directed & 24 \\ \cline{ 2- 7}
%  & twitter & 41,652,230 & 2,405,026,092 & 22,885 MB & Directed & 35 \\ \hline
% Syn & kron21 & 2,097,152 & 182,084,020 & 1,471 MB & Undirected & 87 \\ \hline
% \end{tabular}
% \label{tab:graphs}
% \end{center}
% \end{footnotesize}
% \end{table}

\begin{table}[t]
\begin{threeparttable}
\begin{scriptsize}
 \caption{Graph data used for experiments}
 \label{tab:iter}
 \centering
 %\begin{tabular}{|c|l|r|r|r|r|r|}
\begin{tabular}{lrrrrr}
     \toprule
\multicolumn{1}{l}{Name} & \multicolumn{1}{c}{$|V|$} &
\multicolumn{1}{c}{$|E|$ $^a$} & %\footnote{Sorted by the number of edges}} &
\multicolumn{1}{c}{Size} & \multicolumn{1}{c}{Structure} &
\multicolumn{1}{c}{Deg $^b$} \\ %\footnote{The average degree of each graph}} \\
\midrule
cit\_Patents & 4E+06 & 17E+06 & 249 MB & Directed & 24 \\
Soc-LiveJournal1 & 5E+06 & 69E+06 & 964 MB & Directed & 14 \\
Soc-twitter-2010 & 19E+06 & 298E+06 & 4,809 MB & Undirected & 16 \\
uk-2002 & 21E+06 & 530E+06 & 4,977 MB & Directed & 25 \\
uk-2005 & 40E+06 & 936E+06 & 15,689 MB & Directed & 24 \\
twitter & 42E+06 & 2,405E+06 & 22,885 MB & Directed & 35 \\
kron21 $^c$ & 2E+06 & 182E+06 & 1,471 MB & Undirected & 87 \\  %\footnote{Synthetic graph generated by Kronecker graph generator}
    \bottomrule
    \end{tabular}
    \begin{tablenotes}
      \tiny
      \item {$^a$ The average degree of each graph, $^b$ Sorted by the number of edges, $^c$ Synthetic graph}
    \end{tablenotes}
 \label{tab:graphs}
\end{scriptsize}
\end{threeparttable}
\end{table}


%see Table \ref{tab:graph-loading} for details on how the Hornet and Gunrock frameworks load graph data.
% Oded->Euna, you wanted to add some details on how some of the frameworks were dealing with undirected vs. directed internally when they were loading the inputs.
% [Euna] made it as a table and footnotes

% \begin{table}[htp!]
% \begin{threeparttable}
%   \caption{Loading Directed or Undirected Graphs}
%   \label{tab:graph-loading}
%   \centering
%   \begin{tabular}{llll}
%   \toprule
%   Graph Type & Loading Parameter & Gunrock & Hornet \\
%   \midrule
%   Directed & --directed & directed & directed \\
%             & --undirected\tnote{1} & undirected & undirected \\
%   \midrule         
%   Undirected & --directed & undirected\tnote{2} & directed\tnote{3} \\
%             & --undirected & undirected & undirected \\
%   \bottomrule
%   \end{tabular}
%       \begin{tablenotes}
%       \item[1] Gunrock and Hornet both load a directed graph as an undirected graph by duplicating edges in the opposite direction.
%       \item[2] Gunrock ignores "--directed" for loading undirected graphs.
%       \item[3] Hornet generates a directed graph from an undirected graph by taking the first read edge and ignoring the second edge in the opposite direction.
%       \end{tablenotes}
% \end{threeparttable}
% \end{table}
%(data type/ memory footprint/ GPU utilization/ energy consumption?)
