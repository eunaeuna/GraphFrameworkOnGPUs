\label{sec:gunrock}

Gunrock~\cite{wang2017gunrock} is a high-performance GPU framework for graph computation. Gunrock's programming model is built around the idea of a \emph{frontier}, a data structure that represents a subset of vertices or edges in the graph. Graph primitives in Gunrock are constructed from \emph{operators} that input one or more frontiers and output one or more frontiers. The most performance-critical operator is \emph{advance}, which inputs a frontier of edges or vertices and outputs a neighboring frontier of edges or vertices. For instance, in a breadth-first search, a vertex-to-vertex advance inputs a frontier of vertices and outputs the frontier of vertices that are connected to vertices in the input frontier. Gunrock also supports \emph{filter} (an operator that uses a user-defined predicate to select a subset of its input frontier), \emph{compute} (an operator that applies a user-defined function to each element in a frontier), \emph{intersection} (an operator that outputs elements contained in both its input frontiers), and \emph{neighbor-reduce} (an operator that outputs a reduction per input element, computed from all of that input element's neighbors). Gunrock's high performance is primarily due to its sophisticated load-balance strategies integrated into its operators. Gunrock's latest release contains 26 graph primitives, including both the primitives characterized in this paper as well as more sophisticated primitives largely taken from DARPA HIVE workloads.
