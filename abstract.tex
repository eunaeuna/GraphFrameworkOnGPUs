\begin{abstract}
Graph frameworks are a great way to abstract both the graph data structure and the algorithmic programming model to enable productivity and high performance on a wide range of systems, including shared-memory, heterogeneous systems, distributed systems, or accelerators. Over the last two decades, a plethora of such systems have been created with the introduction of new hardware.
The introduction of CUDA as a parallel programming language for the NVIDIA GPU roughly a decade ago also created a wider interest in parallel graph algorithms and frameworks. In the last five years, a wide range of algorithms have been designed for the GPU, though many of these are standalone codes that can only deal with one problem or problem type. However, a few frameworks have also been put together to enable portability, productivity, scalability, and performance. These frameworks greatly vary in the exact tools and knowledge needed to develop additional algorithms.

In this paper we explore several of the best performing GPU-based graph frameworks on six NVIDIA GPU systems. While we do compare the performance of the frameworks across several algorithms and input graphs, our goal is not to find and declare one framework to be the winner over the others as we understand that each framework was designed with different objectives and across a different time-span. Rather, our goal is to share our practical experiences with developing both the frameworks and graph algorithms for these frameworks, as well as performance differences in different system configurations, and to offer readers a set of good practices when trying to achieve high performance on modern GPU frameworks.

% [Euna] We discuss 
% 1. execution time
% 2. scalability 
% 3. memory footprint?
% 4. energy consumption?

\end{abstract}



